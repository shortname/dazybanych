\documentclass[a4paper, 11pt]{article}
\usepackage[utf8]{inputenc}
\usepackage{polski}
\usepackage[polish]{babel}
\usepackage{hyperref}
\hypersetup{colorlinks=true, urlcolor=red}

\title{Konfiguracja środowiska}
\author{Jakub Dymon}
\date{10.03.2016}

\begin{document}
	\maketitle
	\section{OSTRZEŻENIE!!!}
	\begin{Large}
		DOKUMENT NIE JEST JESZCZE GOTOWY! DAM WAM ZNAĆ KIEDY BĘDZIE!
	\end{Large}
	\section{Do pobrania}
	\begin{enumerate}
		\item XAMPP (\url{https://www.apachefriends.org/pl/index.html})
		\item GitHub (\url{https://desktop.github.com/})
		\item NetBeans \textbf{dla PHP} {\url{https://netbeans.org/downloads/}}
	\end{enumerate}
	\section{XAMPP}
	\begin{enumerate}
		\item Zainstaluj XAMPP`a
		\item Otwórz \quotedblbase XAMPP Control Panel\textquotedblright
		\item W linijce \quotedblbase Apache\textquotedblright wybierz \quotedblbase Start\textquotedblright
		\item Wejdź do przeglądarki internetowej
		\item Wejdź na \url{http://localhost} (ewentualnie \url{http://localhost:80}). Jeżeli zobaczyłeś stronę internetową, to znaczy, że się udało.
	\end{enumerate}
	\section{GitHub}
	\begin{enumerate}
		\item Załóż konto na \url{github.com}
		\item Zostań dodany do projektu jako \emph{contributor}
		\item Zainstaluj GitHuba i zaloguj się w nim
		\item Wybierz  \quotedblbase +\textquotedblright w lewym górnym rogu okna, dalej \quotedblbase Clone\textquotedblright, \quotedblbase shortname\textquotedblright, \quotedblbase dazybanych\textquotedblright
		\item Wybierz lokalizację
		\item Znajdź w folderze, gdzie zainstalowałeś XAMPP`a plik apache\textbackslash conf\textbackslash extra\textbackslash httpd-vhosts.conf
		\item Wklej na jego końcu kod, który znajdziesz w plikach z repo na ścieżce \textbackslash Instrukcje\textbackslash wklejka.txt (podmień XXX na ścieżkę do folderu Kod\textbackslash dazybanych w plikach z repo)
		\item Jeśli jeszcze tego nie zrobiłeś, uruchom serwer \quotedblbase Apache\textquotedblright z poziomu \quotedblbase XAMPP Control Panel\textquotedblright
		\item Wejdź do przeglądarki internetowej
		\item Wejdź na \url{http://localhost}. Jeżeli zobaczyłeś stronę \quotedblbase Hello world!\textquotedblright, to znaczy, że się udało
	\end{enumerate}
	\section{NetBeans}
	\begin{enumerate}
		\item Zainstaluj NetBeans`a
		\item Otwórz program i wybierz \quotedblbase Open project\textquotedblright
		\item Wybierz w plikach z repo folder Kod\textbackslash dazybanych
		\item Naciśnij F6 albo zieloną strzałkę w górnej części okna
		\item Jeżeli zobaczyłeś \quotedblbase Hello world!\textquotedblright, to znaczy, że się udało 
	\end{enumerate}
	\section{MySQL Server}
	\begin{enumerate}
		\item Jeżeli jeszcze tego nie zrobiłeś, zainstaluj XAMPP`a
		\item Otwórz \quotedblbase XAMPP Control Panel\textquotedblright
		\item W linijce \quotedblbase MySQL\textquotedblright wybierz \quotedblbase Start\textquotedblright
		\item Otwórz przeglądarkę internetową i wejdź na \url{http://localhost/phpmyadmin}. Jeżeli zobaczyłeś panel adminstracyjny bazy danych, to znaczy, że baza działa.
		\item Wejdź w zakładkę \quotedblbase User accounts\textquotedblright, zaznacz wszystkich użytkowników poza \quotedblbase root\textquotedblright na \quotedblbase localhost\textquotedblright i kliknij \quotedblbase Wykonaj\textquotedblright u dołu strony.
		\item Po usunięciu pozostałych użytkowników kliknij \quotedblbase Edit privileges\textquotedblright przy pozostałym użytkowniku, wejdź w taba \quotedblbase Zmień hasło\textquotedblright, jako hasło podaj \quotedblbase admin\textquotedblright i kliknij \quotedblbase Wykonaj\textquotedblright. Po ręcznym odświeżeniu strony powinieneś zobaczyć wielki czerwony komunikat o błędzie: \texttt{\quotedblbase Nie udało się nawiązać połączenia: błędne ustawienia.\textquotedblright}.
		\item Znajdź w folderze, gdzie zainstalowałeś XAMPP`a plik phpMyAdmin\textbackslash config.inc.php
		\item W znalezionym pliku jako zmień wartość przypisania:\\
			\$cfg['Servers'][\$i]['password'] = '';\\
			na 'admin'
		\item Po odświeżeniu strony powinienieś na powrót zobaczyć panel administracyjny bazy danych
		\item Otwórz NetBeans`a
		\item W panelu \quotedblbase Projects\textquotedblright wybierz tab \quotedblbase Services\textquotedblright
		\item Rozwiń \quotedblbase Databases\textquotedblright.
		\item Na liście powinien się znaleźć \quotedblbase MySQL Server at localhost...\textquotedblright. Kliknij PPM i wybierz \quotedblbase Connect\textquotedblright (login: \quotedblbase root\textquotedblright, hasło: \quotedblbase admin\textquotedblright), a potem (ponownie PPM) \quotedblbase Create Database\textquotedblright. Jako nazwę podaj \quotedblbase sklep{\_}sportowy\textquotedblright. 
		\item Wybierz \quotedblbase sklep{\_}sportowy\textquotedblright PPM, a następnie kliknij \quotedblbase Connect\textquotedblright
		\item Jako login i hasło podaj kolejno \quotedblbase root\textquotedblright i \quotedblbase admin\textquotedblright. Na tej samej liście w bocznym panelu trochę niżej powinno się pojawić z kwadratową ikoną \quotedblbase jdbc:mysql{\_}localhost...\textquotedblright.
		\item W NetBeans`ie wybierz \quotedblbase File\textquotedblright, dalej \quotedblbase Open File...\textquotedblright. Spośród plików z repo wybierz Baza Danych\textbackslash sklep{\_}sportowy.sql
		\item W panelu edycji z listy \quotedblbase Connection\textquotedblright wybierz to, zawierające \quotedblbase sklep{\_}sportowy\textquotedblright w nazwie, a następnie wykonaj skrypt (kliknij pierwszą ikonkę na prawo od listy - \quotedblbase Run SQL\textquotedblright)
		\item Zarówno w NetBeans w bocznym panelu, jak i w phpMyAdmin po odświeżeniu powinny się pojawić dwie tabele: \quotedblbase users\textquotedblright i \quotedblbase promocje\textquotedblright
		\item Otwórz przeglądarkę i wejdź na \url{http://localhost/login.php}
		\item Po zalogowaniu (\quotedblbase admin\textquotedblright, \quotedblbase admin\textquotedblright) powinieneś zobaczyć listę towarów. W przypadku podanych błędnych danych logowania powinieneś zostać wrócony na stronę logowania.
	\end{enumerate}
	\section{Uwagi techniczne}
	\begin{itemize}
		\item Z poziomu NetBeans`a można komitować/pulować/puszować pliki, ale tylko z folderu Kod, pozostałe trzeba GitHub`em
	\end{itemize}
\end{document}