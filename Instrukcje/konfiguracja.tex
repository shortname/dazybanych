\documentclass[a4paper, 11pt]{article}
\usepackage[utf8]{inputenc}
\usepackage{polski}
\usepackage[polish]{babel}
\usepackage{hyperref}
\hypersetup{colorlinks=true, urlcolor=red}

\title{Konfiguracja środowiska}
\author{Jakub Dymon}
\date{10.03.2016}

\begin{document}
	\maketitle
	\section{OSTRZEŻENIE!!!}
	\begin{Large}
		DOKUMENT NIE JEST JESZCZE GOTOWY! DAM WAM ZNAĆ KIEDY BĘDZIE!
	\end{Large}
	\section{Do pobrania}
	\begin{enumerate}
		\item XAMPP (\url{https://www.apachefriends.org/pl/index.html})
		\item GitHub (\url{https://desktop.github.com/})
		\item NetBeans \textbf{dla PHP} {\url{https://netbeans.org/downloads/}}
	\end{enumerate}
	\section{XAMPP}
	\begin{enumerate}
		\item Zainstaluj XAMPP`a
		\item Otwórz \quotedblbase XAMPP Control Panel\textquotedblright
		\item W linijce \quotedblbase Apache\textquotedblright wybierz \quotedblbase Start\textquotedblright
		\item Wejdź do przeglądarki internetowej
		\item Wejdź na \url{localhost} (ewentualnie \url{localhost:80}). Jeżeli zobaczyłeś stronę internetową, to znaczy, że się udało.
	\end{enumerate}
	\section{GitHub}
	\begin{enumerate}
		\item Załóż konto na \url{github.com}
		\item Zostań dodany do projektu jako \emph{contributor}
		\item Zainstaluj GitHuba i zaloguj się w nim
		\item Wybierz  \quotedblbase +\textquotedblright w lewym górnym rogu okna, dalej \quotedblbase Clone\textquotedblright, \quotedblbase shortname\textquotedblright, \quotedblbase dazybanych\textquotedblright
		\item Wybierz lokalizację
		\item Znajdź w folderze, gdzie zainstalowałeś XAMPP`a plik apache\textbackslash conf\textbackslash extra\textbackslash httpd-vhosts.conf
		\item Wklej na jego końcu kod, który znajdziesz w plikach z repo na ścieżce \textbackslash Instrukcje\textbackslash wklejka.txt (podmień XXX na ścieżkę do folderu Kod\textbackslash dazybanych w plikach z repo)
		\item Jeśli jeszcze tego nie zrobiłeś, uruchom serwer \quotedblbase Apache\textquotedblright z poziomu \quotedblbase XAMPP Control Panel\textquotedblright
		\item Wejdź do przeglądarki internetowej
		\item Wejdź na \url{http://localhost}. Jeżeli zobaczyłeś stronę \quotedblbase Hello world!\textquotedblright, to znaczy, że się udało
	\end{enumerate}
	\section{NetBeans}
	\begin{enumerate}
		\item Zainstaluj NetBeans`a
		\item Otwórz program i wybierz \quotedblbase Open project\textquotedblright
		\item Wybierz w plikach z repo folder Kod\textbackslash dazybanych
		\item Naciśnij F6 albo zieloną strzałkę w górnej części okna
		\item Jeżeli zobaczyłeś \quotedblbase Hello world!\textquotedblright, to znaczy, że się udało 
	\end{enumerate}
	\section{Uwagi techniczne}
	\begin{itemize}
		\item Z poziomu NetBeans`a można komitować/pulować/puszować pliki, ale tylko z folderu Kod, pozostałe trzeba GitHub`em
	\end{itemize}
\end{document}