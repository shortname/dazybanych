\documentclass[a4paper, 11pt]{article}
\usepackage[utf8]{inputenc}
\usepackage{polski}
\usepackage[polish]{babel}


\title{Opis zadania projektowego}
\author{Tomasz Bartos, Jakub Dymon, Wiktor Gerstenstein}
\date{10 marca 2016}

\begin{document}
\maketitle
\section{Temat i cel projektu}
Temat: "Internetowy sklep sportowy"\\
Cel projektu: projekt oraz implementacja aplikacji webowej imitującej sklep internetowy zarówno od strony klienta jak i sprzedawcy
\section{Opis działania i funkcje systemu}
System będzie pozwalał sprzedawcy na dodawanie towarów do sklepu, wystawianie ich do sprzedaży, kontrolę ilości towaru dostępnej na magazynie oraz generowanie prostych raportów na temat sprzedaży, natomiast kupującemu możliwość wyszukiwania towaru, jego zakupu i listowania dokonanych zakupów oraz statusu zamówienia.\\
Aplikacja będzie dostępna w formie strony internetowej umieszczonej na serwerze i dostępnej po zalogowaniu. Będzie istniała możliwość założenia konta w dwóch wariantach - handlowca i klienta - i zależnie od rodzaju konta będzie użytkownikowi udostępniana określona wersja serwisu.
\section{Założenia architektoniczne przyjęte podczas realizacji systemu}
\section{Wykorzystywane technologie, narzędzia projektowania oraz implementacji systemu}
Technologie:
\begin{itemize}
	\item PHP 7.0
	\item MySQL 5.7.10
	\item HTML 5
	\item CSS 3
\end{itemize}
Narzędzia projektowania:
\begin{itemize}
	\item %Coś do rysowania schematów - nie znam nic poza Dia...
		%Słyszałem w pracy o MagicDraw, ale jest płatne
\end{itemize}
Narzędzia implementacji systemu:
\begin{itemize}
	\item %Coś do pisania w php (testuję Atom, dawno temu używałem Notepad++, ale to tylko podświetlanie składni - bez podpowiedzi); jest jeszcze IntelliJ - używałem do Javy i jest zajebisty, podobno istnieje też odmiana do PHP; na licencji studenckiej jest darmowy
	\item %Coś do pisania w SQL
	\item phpMyAdmin 4.5.5.1 %Łatwo znaleźć hosting z dostępnym phpMyAdmin
	\item XAMPP 5.6.19 %Lokalny serwer php, żeby można było sobie lokalnie grzebać zamiast ciągle wrzucać na serwer - to jedyny, o którym słyszałem i po kursie SCR umiem zainstalować :-P
\end{itemize}
\section{Schemat komunikacji, struktura systemu}
\section{Literatura}
\end{document}