\documentclass[a4paper, 11pt]{article}
\usepackage[utf8]{inputenc}
\usepackage{polski}
\usepackage[polish]{babel}


\title{Opis zadania projektowego}
\author{Tomasz Bartos, Jakub Dymon, Wiktor Gerstenstein}
\date{10 marca 2016}

\begin{document}
\maketitle
\section{Temat i cel projektu}
Temat: "Internetowy sklep sportowy"\\
Cel projektu: projekt oraz implementacja aplikacji webowej imitującej sklep internetowy zarówno od strony klienta jak i sprzedawcy
\section{Opis działania i funkcje systemu}
System będzie pozwalał sprzedawcy na dodawanie towarów do sklepu, wystawianie ich do sprzedaży, kontrolę ilości towaru dostępnej na magazynie oraz generowanie prostych raportów na temat sprzedaży, natomiast kupującemu możliwość wyszukiwania towaru, jego zakupu i listowania dokonanych zakupów oraz statusu zamówienia.\\
Aplikacja będzie dostępna w formie strony internetowej umieszczonej na serwerze i dostępnej po zalogowaniu. Będzie istniała możliwość założenia konta w dwóch wariantach - handlowca i klienta - i zależnie od rodzaju konta będzie użytkownikowi udostępniana określona wersja serwisu.
\section{Założenia architektoniczne przyjęte podczas realizacji systemu}
\section{Wykorzystywane technologie, narzędzia projektowania oraz implementacji systemu}
\section{Schemat komunikacji, struktura systemu}
\section{Literatura}
\end{document}